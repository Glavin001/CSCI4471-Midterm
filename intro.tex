\documentclass[12pt]{article}

\usepackage{amsmath}    % need for subequations
\usepackage{graphicx}   % need for figures
\usepackage{verbatim}   % useful for program listings
\usepackage{color}      % use if color is used in text
\usepackage{subfigure}  % use for side-by-side figures
\usepackage{hyperref}   % use for hypertext links, including those to external documents and URLs

% don't need the following. simply use defaults
\setlength{\baselineskip}{16.0pt}    % 16 pt usual spacing between lines

\setlength{\parskip}{3pt plus 2pt}
\setlength{\parindent}{20pt}
\setlength{\oddsidemargin}{0.5cm}
\setlength{\evensidemargin}{0.5cm}
\setlength{\marginparsep}{0.75cm}
\setlength{\marginparwidth}{2.5cm}
\setlength{\marginparpush}{1.0cm}
\setlength{\textwidth}{150mm}

\begin{comment}
\pagestyle{empty} % use if page numbers not wanted
\end{comment}

% above is the preamble

\begin{document}

\begin{center}
{\large Midterm 3} \\ % \\ = new line
\copyright 2015 by Glavin Wiechert \\
Updated April 4, 2015
\end{center}

\section{3. Perspective Projection}

We want to find the 3x3 (homogenous 2D) matrix that 
computes the shadow of point $P$ on the x-axis assuming 
a light positioned at point $Q$ 
(See Figure 2. Perspective/Shadow: Version 3).

\section{Variables}

$Q = (d, h)$

$P = (x,y)$

$P' = (x', y')$

Assuming that we know points $P$ and $Q$ (values $d$, $h$, $x$, $y$).

\section{Equations}

Let's setup our system of equations.

$x' = Ax + By + C$

$y' = Ax + Ey + F$

$1  = Gx + Hy + I$

Note that the last equation is equal to $1$ because 
since this is a homogeneous matrix we know that the 
last value must be $1$, representing a point.

Let's initialize our 3x3 (homogenous 2D) matrix from the 
coefficients from system of equations above:

\begin{align}
\textbf{M} &=
\begin{pmatrix}
A \hfill & B & C \\
D & E & F \hfill \\
G & H & I \hfill 
\end{pmatrix} \nonumber \\
\end{align}

The vectors are:

\begin{align}
\textbf{P} &=
\begin{pmatrix}
x \\
y \\
1  
\end{pmatrix} \nonumber 
, 
\textbf{P'} &=
\begin{pmatrix}
x' \\
y' \\
1  
\end{pmatrix} \nonumber \\
\end{align}

Which results in the following equation:

\begin{align}
\textbf{P'} &= \textbf{M} \textbf{P}
\end{align}

and expands to the following matrix multiplication equation:

\begin{align}
\begin{pmatrix}
x' \\
y' \\
1  
\end{pmatrix} \nonumber
&=
\begin{pmatrix}
A \hfill & B & C \\
D & E & F \hfill \\
G & H & I \hfill 
\end{pmatrix} \nonumber
\begin{pmatrix}
x \\
y \\
1  
\end{pmatrix} \nonumber \\
\end{align}


The goal is to find the shadow point that goes from $Q$ through $P$ 
and intersects with $P'$, which lies on the $x$ axis. 
From this we can conclude that $y'$ ($y$ of $P'$) is equal to 0 (zero).

Therefore we can set the second row to $0$:

\begin{align}
\textbf{M} &=
\begin{pmatrix}
A \hfill & B & C \\
0 & 0 & 0 \hfill \\
G & H & I \hfill 
\end{pmatrix} \nonumber \\
\end{align}

We also know that the third row must equal $1$ to indicate a homogeneous point.
So we can set the coefficients of $x$ and $y$ to $0$ and coefficient of $1$ in the vector to $1$:

\begin{align}
\textbf{M} &=
\begin{pmatrix}
A \hfill & B & C \\
0 & 0 & 0 \hfill \\
0 & 0 & 1 \hfill 
\end{pmatrix} \nonumber \\
\end{align}

We are trying to find $x'$ ($x$ of $P'$), which is $x$ when $y = 0$.

Let's consider that we start at point $P$, with height $y$. 
We want to reach the $x-axis$ or when $y' = 0$.

$y' = 0 = y - Y$

Where we are trying to find $Y$ such that this equation is satisfied.

We can see that $Y = y$.

$y' = 0 = y - y$

However, this is not very helpful, yet.
We are not yet considering the point $Q$ and the slope, 
or rate of change of the line.

We want something like ${\frac{y}{h}}$ that incorporates both point $Q$ and $P$
(note that $h$ is the height or $y$-like value of $Q$).

Let's multiple the second $y$ (or $Y$) by ${\frac{h}{h}}$:

$y' = 0 = y - y {\frac{h}{h}}$

And let's rearrange:

$y' = 0 = y - {\frac{y}{h}} h$

Notice that the equation is still valid and we can now see ${\frac{y}{h}}$ as a coefficient of $h$.

$y' = 0 = y - m h$

We can consider ${\frac{y}{h}}$ the change required to make 
point $P$ intersect with the $x-axis$ or ($y = 0$)

Let's take that and plug it into an equation solving for $x'$.
Similarly to $y'$, the equation for $x'$ would be:

$x' = x - m d$

And substituting $m$ for ${\frac{y}{h}}$:

$x' = x - {\frac{y}{h}} d$

We can then rearrange and obtain our desired form that we used for our matrix:

$V = (A)x + (B)y + (C)1$

In rearranged form:

$x' = (1)x + ({\frac{-d}{h}})y = (0)1$

Therefore our coefficients are:

$A = 1$

$B = {\frac{-d}{h}}$

$C = 0$

Which gives us the final matrix:

\begin{align}
\textbf{M} &=
\begin{pmatrix}
1 & {\frac{-d}{h}} & 0 \\
0 & 0 & 0 \hfill \\
0 & 0 & 1 \hfill 
\end{pmatrix} \nonumber \\
\end{align}

\end{document}
